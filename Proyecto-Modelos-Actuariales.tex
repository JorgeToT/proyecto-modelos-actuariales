% Options for packages loaded elsewhere
\PassOptionsToPackage{unicode}{hyperref}
\PassOptionsToPackage{hyphens}{url}
%
\documentclass[
]{article}
\usepackage{amsmath,amssymb}
\usepackage{lmodern}
\usepackage{iftex}
\ifPDFTeX
  \usepackage[T1]{fontenc}
  \usepackage[utf8]{inputenc}
  \usepackage{textcomp} % provide euro and other symbols
\else % if luatex or xetex
  \usepackage{unicode-math}
  \defaultfontfeatures{Scale=MatchLowercase}
  \defaultfontfeatures[\rmfamily]{Ligatures=TeX,Scale=1}
\fi
% Use upquote if available, for straight quotes in verbatim environments
\IfFileExists{upquote.sty}{\usepackage{upquote}}{}
\IfFileExists{microtype.sty}{% use microtype if available
  \usepackage[]{microtype}
  \UseMicrotypeSet[protrusion]{basicmath} % disable protrusion for tt fonts
}{}
\makeatletter
\@ifundefined{KOMAClassName}{% if non-KOMA class
  \IfFileExists{parskip.sty}{%
    \usepackage{parskip}
  }{% else
    \setlength{\parindent}{0pt}
    \setlength{\parskip}{6pt plus 2pt minus 1pt}}
}{% if KOMA class
  \KOMAoptions{parskip=half}}
\makeatother
\usepackage{xcolor}
\usepackage[margin=1in]{geometry}
\usepackage{color}
\usepackage{fancyvrb}
\newcommand{\VerbBar}{|}
\newcommand{\VERB}{\Verb[commandchars=\\\{\}]}
\DefineVerbatimEnvironment{Highlighting}{Verbatim}{commandchars=\\\{\}}
% Add ',fontsize=\small' for more characters per line
\usepackage{framed}
\definecolor{shadecolor}{RGB}{248,248,248}
\newenvironment{Shaded}{\begin{snugshade}}{\end{snugshade}}
\newcommand{\AlertTok}[1]{\textcolor[rgb]{0.94,0.16,0.16}{#1}}
\newcommand{\AnnotationTok}[1]{\textcolor[rgb]{0.56,0.35,0.01}{\textbf{\textit{#1}}}}
\newcommand{\AttributeTok}[1]{\textcolor[rgb]{0.77,0.63,0.00}{#1}}
\newcommand{\BaseNTok}[1]{\textcolor[rgb]{0.00,0.00,0.81}{#1}}
\newcommand{\BuiltInTok}[1]{#1}
\newcommand{\CharTok}[1]{\textcolor[rgb]{0.31,0.60,0.02}{#1}}
\newcommand{\CommentTok}[1]{\textcolor[rgb]{0.56,0.35,0.01}{\textit{#1}}}
\newcommand{\CommentVarTok}[1]{\textcolor[rgb]{0.56,0.35,0.01}{\textbf{\textit{#1}}}}
\newcommand{\ConstantTok}[1]{\textcolor[rgb]{0.00,0.00,0.00}{#1}}
\newcommand{\ControlFlowTok}[1]{\textcolor[rgb]{0.13,0.29,0.53}{\textbf{#1}}}
\newcommand{\DataTypeTok}[1]{\textcolor[rgb]{0.13,0.29,0.53}{#1}}
\newcommand{\DecValTok}[1]{\textcolor[rgb]{0.00,0.00,0.81}{#1}}
\newcommand{\DocumentationTok}[1]{\textcolor[rgb]{0.56,0.35,0.01}{\textbf{\textit{#1}}}}
\newcommand{\ErrorTok}[1]{\textcolor[rgb]{0.64,0.00,0.00}{\textbf{#1}}}
\newcommand{\ExtensionTok}[1]{#1}
\newcommand{\FloatTok}[1]{\textcolor[rgb]{0.00,0.00,0.81}{#1}}
\newcommand{\FunctionTok}[1]{\textcolor[rgb]{0.00,0.00,0.00}{#1}}
\newcommand{\ImportTok}[1]{#1}
\newcommand{\InformationTok}[1]{\textcolor[rgb]{0.56,0.35,0.01}{\textbf{\textit{#1}}}}
\newcommand{\KeywordTok}[1]{\textcolor[rgb]{0.13,0.29,0.53}{\textbf{#1}}}
\newcommand{\NormalTok}[1]{#1}
\newcommand{\OperatorTok}[1]{\textcolor[rgb]{0.81,0.36,0.00}{\textbf{#1}}}
\newcommand{\OtherTok}[1]{\textcolor[rgb]{0.56,0.35,0.01}{#1}}
\newcommand{\PreprocessorTok}[1]{\textcolor[rgb]{0.56,0.35,0.01}{\textit{#1}}}
\newcommand{\RegionMarkerTok}[1]{#1}
\newcommand{\SpecialCharTok}[1]{\textcolor[rgb]{0.00,0.00,0.00}{#1}}
\newcommand{\SpecialStringTok}[1]{\textcolor[rgb]{0.31,0.60,0.02}{#1}}
\newcommand{\StringTok}[1]{\textcolor[rgb]{0.31,0.60,0.02}{#1}}
\newcommand{\VariableTok}[1]{\textcolor[rgb]{0.00,0.00,0.00}{#1}}
\newcommand{\VerbatimStringTok}[1]{\textcolor[rgb]{0.31,0.60,0.02}{#1}}
\newcommand{\WarningTok}[1]{\textcolor[rgb]{0.56,0.35,0.01}{\textbf{\textit{#1}}}}
\usepackage{graphicx}
\makeatletter
\def\maxwidth{\ifdim\Gin@nat@width>\linewidth\linewidth\else\Gin@nat@width\fi}
\def\maxheight{\ifdim\Gin@nat@height>\textheight\textheight\else\Gin@nat@height\fi}
\makeatother
% Scale images if necessary, so that they will not overflow the page
% margins by default, and it is still possible to overwrite the defaults
% using explicit options in \includegraphics[width, height, ...]{}
\setkeys{Gin}{width=\maxwidth,height=\maxheight,keepaspectratio}
% Set default figure placement to htbp
\makeatletter
\def\fps@figure{htbp}
\makeatother
\setlength{\emergencystretch}{3em} % prevent overfull lines
\providecommand{\tightlist}{%
  \setlength{\itemsep}{0pt}\setlength{\parskip}{0pt}}
\setcounter{secnumdepth}{-\maxdimen} % remove section numbering
\ifLuaTeX
  \usepackage{selnolig}  % disable illegal ligatures
\fi
\IfFileExists{bookmark.sty}{\usepackage{bookmark}}{\usepackage{hyperref}}
\IfFileExists{xurl.sty}{\usepackage{xurl}}{} % add URL line breaks if available
\urlstyle{same} % disable monospaced font for URLs
\hypersetup{
  hidelinks,
  pdfcreator={LaTeX via pandoc}}

\author{}
\date{\vspace{-2.5em}}

\begin{document}

\hypertarget{ajuste-de-distribuciuxf3n-de-probabilidad-a-montos-de-pago-de-siniestros-de-credito-a-la-vivienda-durante-el-auxf1o-del-2020}{%
\section{Ajuste de distribución de probabilidad a montos de pago de
Siniestros de Credito a la Vivienda durante el año del
2020}\label{ajuste-de-distribuciuxf3n-de-probabilidad-a-montos-de-pago-de-siniestros-de-credito-a-la-vivienda-durante-el-auxf1o-del-2020}}

Martha Paola Hernández Soto

Modelos Actuariales

Jorge Aguilar Velázquez

2022-11-18

\hypertarget{introducciuxf3n.}{%
\subsection{Introducción.}\label{introducciuxf3n.}}

\hypertarget{marco-teuxf3rico.}{%
\subsection{Marco Teórico.}\label{marco-teuxf3rico.}}

\hypertarget{metodologuxeda.}{%
\subsection{Metodología.}\label{metodologuxeda.}}

\hypertarget{proceso-de-ajuste.}{%
\subsection{Proceso de Ajuste.}\label{proceso-de-ajuste.}}

\begin{Shaded}
\begin{Highlighting}[]
\FunctionTok{setwd}\NormalTok{(}\StringTok{"\textasciitilde{}/Workspace Universidad/Proyecto{-}Modelos"}\NormalTok{)}

\FunctionTok{library}\NormalTok{(ggplot2)}
\FunctionTok{library}\NormalTok{(actuar)}
\FunctionTok{library}\NormalTok{(survival)}
\FunctionTok{library}\NormalTok{(MASS)}
\FunctionTok{library}\NormalTok{(fitdistrplus)}
\FunctionTok{library}\NormalTok{(moments)}
\FunctionTok{library}\NormalTok{(nortest)}
\FunctionTok{library}\NormalTok{(sandwich)}
\FunctionTok{library}\NormalTok{(momentfit)}
\FunctionTok{library}\NormalTok{(grid)}
\FunctionTok{library}\NormalTok{(vcd)}
\FunctionTok{library}\NormalTok{(fdth)}
\FunctionTok{library}\NormalTok{(openxlsx)}
\FunctionTok{library}\NormalTok{(kdensity)}
\end{Highlighting}
\end{Shaded}

\begin{Shaded}
\begin{Highlighting}[]
\NormalTok{db }\OtherTok{=} \FunctionTok{read.csv}\NormalTok{(}\StringTok{"./db.csv"}\NormalTok{)}

\NormalTok{db\_clean }\OtherTok{=} \FunctionTok{data.frame}\NormalTok{(}\StringTok{"Monto Pagado"}\OtherTok{=}\NormalTok{db}\SpecialCharTok{$}\NormalTok{MONTO.PAGADO}\SpecialCharTok{/}\DecValTok{10000}\NormalTok{)}
\NormalTok{db\_clean }\OtherTok{=}\NormalTok{ db\_clean[}\FunctionTok{c}\NormalTok{(}\DecValTok{1}\SpecialCharTok{:}\DecValTok{100}\NormalTok{),]}

\FunctionTok{print}\NormalTok{(}\FunctionTok{paste}\NormalTok{(}\StringTok{"El número total de datos NA es: "}\NormalTok{,}\FunctionTok{sum}\NormalTok{(}\FunctionTok{is.na}\NormalTok{(db\_clean))))}
\end{Highlighting}
\end{Shaded}

\begin{verbatim}
## [1] "El número total de datos NA es:  0"
\end{verbatim}

\begin{Shaded}
\begin{Highlighting}[]
\FunctionTok{print}\NormalTok{(}\FunctionTok{paste}\NormalTok{(}\StringTok{"El número total de datos iguales a 0 es: "}\NormalTok{,}\FunctionTok{sum}\NormalTok{(db\_clean}\SpecialCharTok{==}\DecValTok{0}\NormalTok{)))}
\end{Highlighting}
\end{Shaded}

\begin{verbatim}
## [1] "El número total de datos iguales a 0 es:  2"
\end{verbatim}

\begin{Shaded}
\begin{Highlighting}[]
\NormalTok{db\_clean }\OtherTok{=} \FunctionTok{data.frame}\NormalTok{(}\StringTok{"Monto Pagado"}\OtherTok{=}\NormalTok{db\_clean)}
\NormalTok{db\_clean }\OtherTok{=}\NormalTok{ db\_clean[db\_clean}\SpecialCharTok{\textgreater{}}\DecValTok{0}\NormalTok{,]}

\NormalTok{data }\OtherTok{=} \FunctionTok{data.frame}\NormalTok{(}\StringTok{"Monto Pagado"}\OtherTok{=}\NormalTok{db\_clean)}
\end{Highlighting}
\end{Shaded}

\begin{Shaded}
\begin{Highlighting}[]
\FunctionTok{summary}\NormalTok{(data}\SpecialCharTok{$}\NormalTok{Monto.Pagado)}
\end{Highlighting}
\end{Shaded}

\begin{verbatim}
##     Min.  1st Qu.   Median     Mean  3rd Qu.     Max. 
##    1.501   23.851   61.305  358.831  277.151 3180.337
\end{verbatim}

\end{document}
